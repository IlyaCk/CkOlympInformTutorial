\Tutorial	Коло --- множина точок, рівновіддалених від центру, тож досліджувана точка потрапляє всер\'{е}дину к\'{о}ла, коли відстань між нею і центром кола менша за радіус, на сам\'{е} коло --- коли рівна, і\nolinebreak[3] назовні, коли більша. Ця відстань рівна $\sqrt{(x_i-x_C)^2+(y_i-y_C)^2}$ (де\nolinebreak[3] $(x_i; y_i)$\nolinebreak[3] --- координати досліджуваної точки, $(x_C; y_C)$\nolinebreak[3] --- центру к\'{о}ла). Тож (для програми, що працює \emph{лише для \mbox{1-го}} способу подання к\'{о}ла) лишається тільки написати цикл з\nolinebreak[3] вкладеними розгалуженнями на три випадки. При бажанні, можна позбутися похибок (див.\nolinebreak[3] стор.~\pageref{sec:floating-point}--\nolinebreak[3]\pageref{text:floating-point-end}), прибравши корені, тобто порівнювати ${(x_i-x_C)^2}\dib{{+}}{(y_i-y_C)^2}$ з~$R^2$. Остаточна реалізація\nolinebreak[3] --- \verb"ideone.com/k9iLan"

Розібратися, що робити з \mbox{2-им} способом подання к\'{о}ла\nolinebreak[3] --- \emph{значно} складніше, можливо навіть складніше усієї решти 
${2\lefteqn{{}^{1}}{}^{\,}\lefteqn{/}{}_{\,\,\,2}}$\nolinebreak[3] задач даного туру. Можна звести \mbox{2-ий} випадок до \mbox{1-го}, тобто перейти від трьох точок до центру й радіусу проведеного через ці три точки к\'{о}ла, та використати вже розглянутий розв'язок.

Один зі\nolinebreak[3] способів\nolinebreak[3] --- будувати перетин серединних перпендикулярів двох сторін $\triangle{}ABC$, тобто ті~ж дії, що й на\nolinebreak[3] уроці геометрії, але виконані замість циркуля і лінійки засобами \emph{обчислювальної геометрії} (\emph{вычислительная геометрия}, \emph{computational geometry}). Вступ до обч.~геометрії можна знайти у\nolinebreak[3] багатьох місцях, зокрема \verb"https://goo.gl/6yppjy"\hspace{0.5em plus 1em} Програма, що розв'язує цим способом\nolinebreak[3] --- \verb"ideone.com/k1aFcF" (але звідти свідомо прибрані допоміжні функції, пояснені за попереднім посиланням).

Інший спосіб\nolinebreak[3] --- вивести на\nolinebreak[3] папері прямі формули, розв'язавши систему
$\left\{
\begin{array}{c}
(x{-}x_A)^2+(y{-}y_A)^2 = (x{-}x_B)^2+(y{-}y_B)^2,\\
(x{-}x_A)^2+(y{-}y_A)^2 = (x{-}x_C)^2+(y{-}y_C)^2,
\end{array}
\right.$
де $x_A$,~$y_A$, $x_B$,~$y_B$, $x_C$,~$y_C$\nolinebreak[3] --- коор\-ди\-нати заданих точок (до\nolinebreak[3] них треба ставитися як до\nolinebreak[3] відомих значень, а\nolinebreak[3] не\nolinebreak[3] змінних), а\nolinebreak[3] $x$\nolinebreak[1] та\nolinebreak[3] $y$\nolinebreak[3] --- координати центра к\'{о}ла,\linebreak[1] їх-то\nolinebreak[1] й шукаємо. Розв'язати цю систему легше, ніж може здатися, бо після перетворень, аналогічних ${(x{-}x_A)^2}\dib{{=}}x^2\dib{{-}}2x_A{\cdot}x\dib{{+}}{x_A}^2$, можна позводити $x^2$ та~$y^2$, і\nolinebreak[3] рівняння стають лінійними відносно $x$ та~$y$ (що логічно, бо це серединні перпендикуляри).

\myflfigaw{$\left|\begin{array}{cccc}
x_A	& y_A	& x_A^2+y_A^2	&	1	\\
x_B	& y_B	& x_B^2+y_B^2	&	1	\\
x_C	& y_C	& x_C^2+y_C^2	&	1	\\
x_i	& y_i	& x_i^2+y_i^2	&	1	\\
\end{array}\right|$}
Можна й не виражати \mbox{2-ий} випадок через \mbox{1-ий}. Виявляється, відповідь на задачу можна взнавати за знаком наведеного праворуч детермінанта. (\emph{Детермінант}, він\nolinebreak[3] же \emph{визначник}\nolinebreak[3] --- стандартний термін теорії матриць. Як\nolinebreak[3] взагалі обчислювати детермінант та чому даний детермінант має таку властивість, бажаючі можуть знайти в\nolinebreak[3] Інтернеті або літературі самостійно.) Так\nolinebreak[3] от: 0\nolinebreak[3] означає, що ${(x_i;y_i)}$ лежить на к\'{о}лі, проведеному через ${(x_A;y_A)}$, ${(x_B;y_B)}$, ${(x_C;y_C)}$; додатне значення детермінанта\nolinebreak[3] --- що всер\'{е}дині; від'ємне\nolinebreak[3] --- ззовні. Тільки це якщо обхід $\triangle{}ABC$ (с\'{а}ме у\nolinebreak[3] порядку $A$, $B$,~$C$) відбувається у\nolinebreak[3] додатному напрямку (проти годинникової стрілки), а\nolinebreak[3] при зміні напрямку обходу $\triangle{}ABC$ слід обміняти місцями смисл додатного і від'ємного знаків детермінанту. Факт зовсім не очевидний, але ж це \emph{лише один із}\nolinebreak[3] способів розв'язання задачі\dots

