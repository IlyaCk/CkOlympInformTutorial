\Tutorial	Оскільки коло --- множина точок, рівновіддалених від центру, то досліджувана точка потрапляє всер\'{е}дину к\'{о}ла, коли відстань між нею і центром кола менша за його радіус, на сам\'{е} коло --- коли рівна радіусу, і\nolinebreak[3] назовні, коли більша. Сам\'{а} ця відстань може бути обчислена як $\sqrt{(x_i-x_C)^2+(y_i-y_C)^2}$ (де\nolinebreak[3] $(x_i; y_i)$\nolinebreak[3] --- координати досліджуваної точки, $(x_C; y_C)$\nolinebreak[3] --- центру к\'{о}ла). Тож (для програми, що працює \emph{лише для \mbox{1-го}} способу подання к\'{о}ла) лишається тільки написати цикл з\nolinebreak[3] розгалуженнями (вкладеними, щоб розібрати три випадки). Доцільно пам'ятати, що при роботі з дійсними числами, через особливості їх зберігання у пам'яті комп'ютера, користуватися оператором "рівно" (Pascal: \verb"=", C++: \verb"==") небезпечно. Один із способів уникнути цього --- намагатися користуватися лише цілими числам --- підходить для цього методу. Тобто замість перевірки $\sqrt{(x_i-x_C)^2+(y_i-y_C)^2}=R$, доцільно робити перевірку $(x_i-x_C)^2+(y_i-y_C)^2=R^2$, де $R$ - радіус кола. Наприклад, див.\nolinebreak[3] \verb"ideone.com/k9iLan"

Розібратися, що робити з \mbox{2-им} способом подання к\'{о}ла\nolinebreak[3] --- \emph{набагато} складніше, можливо навіть складніше усієї решти 
(<<${2\lefteqn{{}^{1}}{}^{\,}\lefteqn{/}{}_{\,\,\,2}}$\nolinebreak[3] задач>>) даного туру. Усі відомі автору задачі способи так чи інакше зводять \mbox{2-ий} випадок до \mbox{1-го}, тобто спочатку переходять від трьох точок до центру й радіусу к\'{о}ла, проведеного через ці три точки, а потім використовують вже розглянутий розв'язок.

Один зі\nolinebreak[3] способів\nolinebreak[3] --- побудувати описане коло\nolinebreak[2] $\triangle{}ABC$ через побудову перетину серединних перпендикулярів сторін, тобто ті~ж дії, що й на\nolinebreak[3] уроці геометрії, але виконані замість циркуля і лінійки інструментами \emph{обчислювальної геометрії} (рос.\nolinebreak[2] <<\emph{вычислительная геометрия}>>, англ.\nolinebreak[2] <<\emph{computational geometry}>>). Вступ до обч.~геометрії можна знайти у\nolinebreak[3] багатьох місцях, зокрема \verb"https://goo.gl/6yppjy"\hspace{0.5em plus 1em} Програму, що розв'язує цим способом, можна бачити за\nolinebreak[3] посиланням \verb"ideone.com/k1aFcF" (але звідти свідомо прибрані допоміжні функції, пояснені за попереднім посиланням).

Можна й вивести (на\nolinebreak[3] папері) прямі формули, наприклад розв'язавши систему
$\left\{
\begin{array}{c}
(x{-}x_A)^2+(y{-}y_A)^2 = (x{-}x_B)^2+(y{-}y_B)^2,\\
(x{-}x_A)^2+(y{-}y_A)^2 = (x{-}x_C)^2+(y{-}y_C)^2,
\end{array}
\right.$
де $x_A$,~$y_A$, $x_B$,~$y_B$, $x_C$,~$y_C$\nolinebreak[3] --- координати заданих точок (тому до\nolinebreak[3] них треба ставитися як до\nolinebreak[3] відомих значень, а\nolinebreak[3] не\nolinebreak[3] як до\nolinebreak[3] змінних), а\nolinebreak[3] $x$\nolinebreak[1] та\nolinebreak[3] $y$\nolinebreak[3] --- координати центра к\'{о}ла,\linebreak[1] от\nolinebreak[1] їх-то\nolinebreak[1] і шукаємо. Розв'язати цю систему легше, ніж може здатися, бо після перетворення ${(x{-}x_A)^2}\dib{{=}}x^2\dib{{-}}2x_A{\cdot}x\dib{{+}}{x_A}^2$ та решти аналогічних можна позводити $x^2$ та~$y^2$, і\nolinebreak[3] система виявляється лінійною відносно $x$ та~$y$.

В~принципі можливі й інші підходи.
