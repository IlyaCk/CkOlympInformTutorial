\section{Передмова}

Даний збірник містить задачі олімпіад з~інформатики (програмування), що відбувалися у Черкаській області у\nolinebreak[2] \mbox{2013/14} та \mbox{2014/15} навчальних роках. Точніше, він охоплює тури обласних інтернет-олімпіад, а також др\'{у}гі (районні/\linebreak[1]міські) та треті (обласні) етапи Всеукраїнської олімпіади з~інформатики. Для кожної задачі наведені умова та вказівки щодо розв'язування.

Даний збірник не~призначений замінювати собою підручник з програмування. По-перше, ідучи від конкретних задач, нереально побудувати збалансований курс, що\nolinebreak[2] розглядає продуманий перелік\nolinebreak[2] тем. По-друге, співвідношення обсягу задач (23\nolinebreak[3] задачі з\nolinebreak[3] 6\nolinebreak[3] турів) та обсягу збірника \ifallIdeOneLinksCopiedHere\else(\pageref{LastPage}~сторінок) \fi{}робить немож\-ливим детальний розгляд усіх потрібних для розв'язання цих задач алгоритмів. Тому основна увага приділена поясненням нестандартних моментів у розв'язуванні цих задач, а\nolinebreak[3] коли задача зводиться до реалізації відомого алгоритму, іноді наведені лише його назва і порада пошукати в\nolinebreak[3] Інтернеті або літературі, іноді суть викладена, але коротко. Тому рекомендується \emph{поєднувати} даний збірник із підручниками з\nolinebreak[3] програмування, монографіями \mbox{та/або} сайтами, де розглядають ефективні алгоритми, тощо.

Тим не менш, розділ~\ref{sec:FAQ}\nolinebreak[3] (стор.~\pageref{sec:FAQ}--\nolinebreak[4]\pageref{text:FAQ-end}) містить огляд деяких питань, які дуже часто потрібні при обговоренні олімпіадних задач, але можуть бути погано відомі тим, хто займався лише іншими сторонами програмування.

Оскільки на олімпіаді з~інформатики (програмування) р\'{о}зв'язком учасника є програма, логічно, щоб даний збірник містив також і\nolinebreak[3] тексти таких програм. І\nolinebreak[3] для більшості задач вони %%%to be continued inside if-else-fi
\ifallIdeOneLinksCopiedHere
включені безпосередньо у\nolinebreak[3] текст даного збірника. Точніше кажучи, лише у дану верстку, яка поширюється лише у\nolinebreak[3] електронному вигляді. С\'{а}ме через ці включення усіх програм дана верстка у деяких місцях некрасива, та й виковирювати ці тексти програм з\nolinebreak[3] даного \texttt{.pdf} не\nolinebreak[3] дуже зручно. Тому, хоч вони й є прямо тут, все\nolinebreak[3] ж спробуйте
\else
доступні. Зокрема, як\nolinebreak[3] посилання на\nolinebreak[3] 
\fi
сайт \verb"ideone.com", де їх можна бачити з\nolinebreak[3] підсвіткою синтаксиса, скачувати (у\nolinebreak[3] вигляді, придатному для редагування і компілювання), а\nolinebreak[3] при бажанні\nolinebreak[3] --- створювати власну копію (ця\nolinebreak[3] дія називається <<fork>>) і\nolinebreak[3] працювати з\nolinebreak[3] нею безпосередньо на\nolinebreak[3] сайті (буває зручно, якщо треба швиденько щось спробувати на чужому комп'ютері, де\nolinebreak[3] нема середовища програмування). 
Крім того, 
\colorbox{yellow}{(десь на сайті cit.ckipo.edu.ua --- змінити, узгодивши з ІВФ)} % TODO 
\colorbox{yellow}{розміщено} і\nolinebreak[3] дану версію збірника, і\nolinebreak[3] версію%
\ifallIdeOneLinksCopiedHere%
\ (яка власне й була надрукована офіційно), що \emph{не}\nolinebreak[3] містить текстів програм (лише посилання на \verb"ideone.com"), і\nolinebreak[3] має краще вивірену верстку.
\else%
, де всі тексти програм вверстані у сам збірник (але через це збільшується об'єм і\nolinebreak[3] погіршується якість верстки).
\fi

Насамкінець, про авторство даного збірника. Близько третини його тексту складають умови задач, які формувалися авторським колективом. Особисто свій внесок в умови задач упорядник збірника Порубльов~І.~М. оцінює приблизно у одну третину, а\nolinebreak[3] решта зроблено такими людьми: Богатирьов~О.~О.\nolinebreak[3] (ЧНУ), Фурник~І.~В.\nolinebreak[3] (ЧОІПОПП), Шемшур~В.~М.\nolinebreak[3] (ЧОІПОПП), Безпальчук~В.~М.\nolinebreak[3] (ЧНУ), Черненко~Р.~В. (програміст, випускник ЧНУ 2011~р.), Поліщук~Д.~І. (фізик та програміст, випускник \mbox{ФіМЛі} 2004~р.). 
Але всі ці інші автори задач зробили лише малу частину розборів (пояснень), переважна більшість ($\approx$70--80\%) тексту яких написана особисто Порубльовим~І.~М., а~крім того є значний внесок Полосухіна~В.~А. (випускник\nolinebreak[2] \mbox{ФіМЛі} 2014~р., брав участь лише у\nolinebreak[3] написанні розборів \emph{після} відповідних турів).