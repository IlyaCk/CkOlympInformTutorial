\vspace{10mm}\par
\begin{problem}{Хмарочоси}{Клавіатура (stdin) або файл input.txt}{Екран (stdout) або файл output.txt}{2 сек}{64 мегабайти}

Як~відомо, місто Прямий\nolinebreak[2] Ріг являє собою одну пряму вулицю, уздовж якої прокладена координатна вісь~$Ox$. Останнім часом у~місті розгорнувся будівельний бум, внаслідок якого ПрямМіськБуд звів $N$\nolinebreak[3] хмарочосів. Кожен хмарочос можна охарактеризувати висотою $h_i$ та координатою $x_i$ (усі $x_i$ різні, усі $h_i$ строго додатні).

За задумами мерії Прямого\nolinebreak[2] Рогу, настав час обладнати на дахах деяких із хмарочосів оглядові майданчики. Прибутковість такого майданчику залежить від того, скільки з даху даного хмарочосу видно інших хмарочосів. Тому Вас попросили написати програму, яка підготує відповідну статистику. Дані про хмарочоси зберігаються у\nolinebreak[3] мерії в\nolinebreak[3] порядку зростання~$x_i$, тож саме в\nolinebreak[3] такому порядку вони і вводитимуться у Вашу програму. Якщо дахи трьох або більше хмарочосів виявляються розміщеними на\nolinebreak[2] одній прямій, ближчі затуляють дальші, і\nolinebreak[3] дальших не~видно.

\InputFile
Перший рядок містить кількість хмарочосів~$N$, наступні\nolinebreak[2] $N$\nolinebreak[3] рядків містять по два цілі числа кожен --- координату\nolinebreak[3] $x_i$ та висоту~$h_i$. Гарантовано, що ${1{\<}N{\<}4321}$, $0\dib{{\<}}x_1\dib{{<}}x_2\dib{{<}}\dots\dib{{<}}x_N\dib{{\<}}10^6$\nolinebreak[3] (мільйон),\linebreak[2] ${1{\<}h_i{\<}10^5}$\nolinebreak[2] (сто\nolinebreak[3] тисяч). Програма може читати вхідні дані хоч з клавіатури, хоч зі вхідного файлу \verb"input.txt" (але\nolinebreak[2] лише з чогось одного, а\nolinebreak[3] не\nolinebreak[2] поперемінно).

\OutputFile
Програма має вивести $N$\nolinebreak[3] рядків, $i$-ий рядок має містити одне ціле число --- кількість інших хмарочосів, які видно з даху хмарочосу $\No\,i$. Програма може виводити результати хоч на\nolinebreak[3] екран, хоч у вихідний текстовий файл\nolinebreak[2] \verb"output.txt" (але лише на/у щось одне, а\nolinebreak[3] не\nolinebreak[2] поперемінно то\nolinebreak[3] туди то~туди).

\begin{minipage}{\textwidth}

\Example

\vspace{-\baselineskip}

\begin{exampleSimpleThree}{4em}{4em}{15em}{}%
\exmp{11
0 4
3 5
5 3
6 4
7 5
8 8
10 5
12 4
14 3
17 1
19 7}{2
5
3
4
3
10
3
4
4
3
5}{\raisebox{-72pt}{\begin{mfpic}[9.6]{0}{19}{0}{8}
\lines{( 0,0),(19,0)}
\lines{( 0,1),(19,1)}
\lines{( 0,2),(19,2)}
\lines{( 0,3),(19,3)}
\lines{( 0,4),(19,4)}
\lines{( 0,5),(19,5)}
\lines{( 0,6),(19,6)}
\lines{( 0,7),(19,7)}
\lines{( 0,8),(19,8)}
\lines{( 0,0),( 0,8)}
\lines{( 1,0),( 1,8)}
\lines{( 2,0),( 2,8)}
\lines{( 3,0),( 3,8)}
\lines{( 4,0),( 4,8)}
\lines{( 5,0),( 5,8)}
\lines{( 6,0),( 6,8)}
\lines{( 7,0),( 7,8)}
\lines{( 8,0),( 8,8)}
\lines{( 9,0),( 9,8)}
\lines{(10,0),(10,8)}
\lines{(11,0),(11,8)}
\lines{(12,0),(12,8)}
\lines{(13,0),(13,8)}
\lines{(14,0),(14,8)}
\lines{(15,0),(15,8)}
\lines{(16,0),(16,8)}
\lines{(17,0),(17,8)}
\lines{(18,0),(18,8)}
\lines{(19,0),(19,8)}
\pen{1mm}
\lines{(0,0),(19,0)}
\lines{(0,0),(0,4)}
\lines{(3,0),(3,5)}
\lines{(5,0),(5,3)}
\lines{(6,0),(6,4)}
\lines{(7,0),(7,5)}
\lines{(8,0),(8,8)}
\lines{(10,0),(10,5)}
\lines{(12,0),(12,4)}
\lines{(14,0),(14,3)}
\lines{(17,0),(17,1)}
\lines{(19,0),(19,7)}
\end{mfpic}}}%
\end{exampleSimpleThree}

\end{minipage}

\end{problem}
