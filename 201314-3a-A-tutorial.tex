{
\hyphenpenalty=400
\Tutorial	
В\nolinebreak[3] принципі, \emph{можна} вивести аналітичну формулу, але\nolinebreak[3] це\nolinebreak[3] потребує знань з\nolinebreak[3] теорії чисел (бажаючі можуть знайти інформацію за ключовими словами <<\emph{теорія чисел}>>, <<\emph{кільце залишків за модулем}>>, <<\emph{мал\'{а} теорема Ферма}>> і застосувати все\nolinebreak[3] це до даної задачі). Але все\nolinebreak[3] це було\nolinebreak[3] \emph{би} доцільним, \emph{якби} довжина ISBN-коду становила не~10~цифр, а,\nolinebreak[3] наприклад,\nolinebreak[3] сотні тисяч, так що розглянутий далі значно простіший і значно більш <<програмістський>> (а\nolinebreak[3] не\nolinebreak[3] <<математичний>>) підхід працював\nolinebreak[3] \emph{би} надто довго.

А\nolinebreak[3] для 10~цифр підходить і значно простіший алгоритм перебору, тобто перепробувати усі 10 варіантів від~0\nolinebreak[1] до~9 і для кожного подивитися, чи\nolinebreak[3] виконується описана в\nolinebreak[3] умові правильність\nolinebreak[3] ISBN. Приклад реалізації такого підходу див.\nolinebreak[1] \verb"ideone.com/g3SbMb"\hspace{0.5em plus 1em} Правда, якби були різні правильні відповіді, ця програма вивела~б усі, хоча зазвичай на~олімпіадах треба виводити будь-яку одну. Насправді це неактуально, бо дана задача не~може мати різні правильні відповіді (хто вивчав питання, згадані у попередньому абзаці, можуть це легко довести, решта можуть або\nolinebreak[3] повірити, або\nolinebreak[3] дописати у розв'язок \verb"break" для обривання роб\'{о}ти після виведення першої відповіді.)

}
