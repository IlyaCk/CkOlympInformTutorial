\begin{problemAllDefault}{ISBN}

ISBN (з~англ.\nolinebreak[2] International Standard Book Number\nolinebreak[3] --- міжнародний стандартний номер книги) універсальний ідентифікаційний номер, що присвоюється книзі або брошурі з\nolinebreak[3] метою їх класифікації. ISBN\nolinebreak[3] призначений для ідентифікації окремих книг або різних видань та є унікальним для кожного видання книги. Даний номер містить десять цифр, перші дев'ять з\nolinebreak[3] яких ідентифікують книгу, а\nolinebreak[3] остання цифра використовується для перевірки коректності всього номеру\nolinebreak[3] ISBN. Для\nolinebreak[3] перевірки\nolinebreak[3] ISBN обчислюється сума добутків цифр на їхній номер, нумерація при цьому починається з крайньої правої цифри. В~результаті має бути отримано число, що без остачі ділиться на~11.

Наприклад: \textbf{0201103311}\nolinebreak[3] --- коректний номер, тому\nolinebreak[3] що ${\mathbf{0}{\*}10}\dib{{+}}{\mathbf{2}{\*}9}\dib{{+}}{\mathbf{0}{\*}8}\dib{{+}}{\mathbf{1}{\*}7}\dib{{+}}{\mathbf{1}{\*}6}\dib{{+}}{\mathbf{0}{\*}5}\dib{{+}}{\mathbf{3}{\*}4}\dib{{+}}{\mathbf{3}{\*}3}\dib{{+}}{\mathbf{1}{\*}2}\dib{{+}}{\mathbf{1}{\*}1}\dib{{=}}55$, що\nolinebreak[3] націло ділиться на~11.

Кожна з~перших дев'яти цифр може приймати значення від 0 до~9. 

Напишіть програму, що читає ISBN\nolinebreak[2] код з\nolinebreak[3] однією пропущеною цифрою (вона буде позначатись як символ\nolinebreak[2] <<~>>\nolinebreak[3] (пробіл)) і виводить значення пропущеної цифри.


\Example
\begin{exampleSimple}{5em}{3em}%
\exmp{020110 311}{3}%
\end{exampleSimple}

\Note	У~справжніх ISBN-номерах в\nolinebreak[3] якості останньої цифри може бути також велика латинська~\texttt{X}, що позначає~10. Але в\nolinebreak[3] цій задачі таких номерів гарантовано не~буде.

\end{problemAllDefault}
