\Tutorial	По-перше, слід у великому тексті умови знайти ті ${\approx}20\%$, які справді потрібні: \textsl{<<Є\nolinebreak[3] текст, який гарантовано отриманий таким чином: взяли число від\nolinebreak[2] 0 до\nolinebreak[2] 1000000, й замінили кожну цифру на дві букви згідно наведеної таблички. Провести зворотнє перетворення цього текста у\nolinebreak[3] число.>>}.

По-друге, все могло\nolinebreak[3] \emph{би} бути вельми складн\'{и}м, \emph{якби} траплялися ситуації, коли одне зі\nolinebreak[2] слів\nolinebreak[3] --- початок іншого (як-то <<7\nolinebreak[3] позначається як \texttt{mis}, 8\nolinebreak[3] --- як\nolinebreak[2] \texttt{misiv}>>). Так що треба відмовитися від ідеї писати універсальну програму, яка могла\nolinebreak[3] би працювати з різними позначеннями цифр, і\nolinebreak[3] ретельно дослідити, якими конкретними, заданими в\nolinebreak[3] умові, словами кодуються цифри. І\nolinebreak[3] побачити, що тут не\nolinebreak[3] лише нема такої ситуації, а\nolinebreak[3] ще й усі ці слова дволітерні. 

Так що задача насправді досить проста. Треба лише зуміти прочитати це у\nolinebreak[3] громіздкій умові. Наприклад, див.\nolinebreak[2] \verb"ideone.com/BNuDlL"\hspace{0.5em plus 1em} Або, ще нахабніше використавши, що вхідні дані \emph{гарантовано} являють собою якесь закодоване число, можна написати щось іще простіше, наприклад \verb"ideone.com/Odw21F"
