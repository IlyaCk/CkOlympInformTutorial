\Tutorial	{
\hyphenpenalty=400
Діяти за принципом \textsl{<<Викреслювати мінімальну цифру>>}\nolinebreak[3] --- \emph{неправильно} (всупереч підступним пр\'{и}\-кла\-дам з~умови, які провокують таку хибну думку.) Наприклад, із\nolinebreak[2] числ\'{а}\nolinebreak[3] 9891 треба викреслити~8 і отримати~991, а\nolinebreak[3] викреслення мінімальної цифри~1 дасть не~максимальне~989.

Оскільки мова йде про досить малі обмеження (кількість цифр~$\<5$, видаляється лише одна), найпростіший для написання правильний розв’язок\nolinebreak[3] --- явно перебрати всі варіанти викреслювання однієї цифри (усе число без\nolinebreak[3] \mbox{1-ої}, усе число без\nolinebreak[3] \mbox{2-ої}, тощо), і~вибрати з~них максимальний. При цьому зручно (не~обов'язково, але\nolinebreak[2] зручно) перевести прочитане число у\nolinebreak[3] рядок (\texttt{string}), і~займатися вилученням цифр у\nolinebreak[3] \texttt{string}-овому поданні. 
Реалізацію див. \verb"ideone.com/jFBIM6"

Правильним є також і розв’язок \textsl{<<Знайти найлівіше місце, де зразу після меншої цифри йде більша, і викреслити меншу с\'{а}ме з\nolinebreak[2] цих двох; якщо жодного такого місця нема (наприклад, у\nolinebreak[3] числі\nolinebreak[2] 97652) --- викреслити останню цифру>>}.

}

