\Tutorial
Ця задача проста, щоб набрати \emph{частину} балів, але набрати повний бал чи значну частину балів не~так~просто. Розв'язок \verb"ideone.com/w8Yu4L" набирає половину балів. 
І~тут є\nolinebreak[2] де помил\'{и}тися й отримати ще\nolinebreak[3] менше\nolinebreak[3] (важливо і\nolinebreak[3] правильно врахувати смисл~$m$, і\nolinebreak[3] взяти тип \texttt{int64}).

При $n$, починаючи з${}\approx3{\cdot}10^6$, сума $1^2\dib{{+}}2^2\dib{{+}}\dots\dib{{+}}n^2$ виходить навіть за межі типу \texttt{int64} (він\nolinebreak[3] же \texttt{long\nolinebreak[3] long}). З~цим можна боротися, застосовуючи найпростіший прийом т.~зв. \emph{модульної арифметики}, а~с\'{а}ме, роблячи додавання чергового\nolinebreak[3] $i^2$ не~як\nolinebreak[3] \verb"s:=s+sqr(int64(i))", а\nolinebreak[2] як\nolinebreak[3] \verb"s:=(s+sqr(int64(i)))mod m". 

Реалізація, де вчасно робиться ``\texttt{\dots~mod~m}'', набирає 150 балів (з~250). 
Тепер критичним стає те, що $\approx{}10^9$ ітерацій циклу (ще\nolinebreak[3] й з громіздкою операцією \texttt{mod}) не~поміщаються в обмеження часу (1~сек).

\MyParagraph{100\%-ий розв'язок $\No\,$1} 
Якщо знати (або вивести під час туру) формулу $1^2\dib{{+}}2^2\dib{{+}}\dots\dib{{+}}n^2\dib{{=}}\frac{n(n+1)(2n+1)}{6}$, можна поєднати її з засобами модульної арифметики і отримати зовсім інший розв'язок \verb"http://ideone.com/g6nSKa"\hspace{0.5em plus 1em} Він взагалі не~містить циклів, тож вкладається у 1~сек з~величезним запасом.


Щоб написати цей розв'язок, треба знати деякі математичні факти та властивості, і це може не~всім подобатися. Але, по-перше, задача має альтернативний розв'язок; по-друге, навіть якби цей розв'язок був єдиним, це не~суперечило~б традиціям Всеукраїнської олімпіади з~інформатики.

У~цьому розв'язку треба писати с\'{а}ме ``\texttt{\dots\nolinebreak[3] mod\nolinebreak[3] \mbox{(6*m)}}'' (а\nolinebreak[2] не\nolinebreak[3] ``\texttt{\dots\nolinebreak[3] mod~m}''), бо такі властивості модульної арифметики: хоча ${(a+b)\bmod p} \dib{{=}} {\bigl((a\bmod p)}\dib{{+}}{(b\bmod p)\bigr)\bmod p}$ --- правильна тотожність, що виконується для всіх $a$, $b$, $p$, і так само для множення, але для ділення виявляється не~так.
Так що кого зацікавило\nolinebreak[3] --- просимо знайти детальніші відомості про модульну арифметику в Інтернеті або літературі.

Щодо того, як можна вивести формулу $1^2\dib{{+}}2^2\dib{{+}}\dots\dib{{+}}n^2\dib{{=}}\frac{n(n+1)(2n+1)}{6}$ самому під час туру --- див., наприклад, \verb"http://dxdy.ru/topic22151.html"\hspace{0.5em plus 1em} Звісно, доцільність витрачання часу туру на подібні виведення формул істотно залежить від умінь конкретного учасника, та від того, чи\nolinebreak[3] має місце ситуація, коли учасник знає, що така формула в~принципі існує, але не~пам'ятає її точно.


\MyParagraph{100\%-ий спосіб $\No\,$2}
Навіть не~знаючи ні\nolinebreak[2] формули зі способу\nolinebreak[3] $\No\,1$, ні\nolinebreak[2] модульної арифметики, можна побудувати (інший) повнобальний розв'язок, користуючись лише базовими, в~межах загальнообов'язкового мінімуму, знаннями математики, а~також спостережливістю та кмітливістю.

Якщо $n < 2{\cdot}10^6$, можна порахувати відповідь <<в~лоб>>, як на початку розбору.

Інакше (при $n \> 2{\cdot}10^6$, враховуючи, що ${m\<10^6}$) вийде, що проміжок від~1 до~$n$ складається з кількох (як\nolinebreak[3] мінімум, двох, як\nolinebreak[3] максимум --- багатьох сотень мільйонів) проміжків від\nolinebreak[2] 1\nolinebreak[2] до~$m$, від\nolinebreak[2] ${m{+}1}$\nolinebreak[2] до~$2m$, від\nolinebreak[2] ${2m{+}1}$\nolinebreak[2] до~$3m$, і~т.~д.

\vspace{0.375\baselineskip}

\mytextandpicture{Розглянемо (праворуч) послідовність очевидних тотожностей для проміжків від\nolinebreak[2] 1\nolinebreak[2] до~$m$ та від\nolinebreak[2] ${m{+}1}$\nolinebreak[2] до~$2m$. У~виразі ${m^2{+}2m{+}1}$ частина ${m^2{+}2m}$ кратна~$m$ і тому не~впливає на\nolinebreak[3] остат\'{о}чну відповідь задачі.}{\begin{tabular}{r@{}c@{}l|rcl}
$1^2$ & $=$ & 1	& $(m{+}1)^2$ & $=$ & $m^2+2m+1$ \\
$2^2$ & $=$ & 4	& $(m{+}2)^2$ & $=$ & $m^2+4m+4$ \\
$3^2$ & $=$ & 9	& $(m{+}3)^2$ & $=$ & $m^2+6m+9$ \\
 & $\vdots$ & & & $\vdots$\\
$m^2$ &$=$&$m^2$& $(m{+}m)^2$ & $=$ & $m^2+2m^2+m^2$ 
\end{tabular}}

Аналогічно не~впливають на відповідь частини  ${m^2{+}4m}$,  ${m^2{+}6m}$, і~т.~д. Тобто, 
${(m{+}1)^2\bmod m}\dib{{=}}{1^2\bmod m}$,
${(m{+}2)^2\bmod m}\dib{{=}}{2^2\bmod m}$,
${(m{+}3)^2\bmod m}\dib{{=}}{3^2\bmod m}$,\nolinebreak[3] \dots, 
а\nolinebreak[2] звідси\nolinebreak[3] --- сума усього др\'{у}гого проміжку $(m{+}1)^2\dib{{+}}(m{+}2)^2\dib{{+}}\dots\dib{{+}}(m{+}m)^2$ має той самий залишок від ділення на~$m$, що й сума усього першого $1^2\dib{{+}}2^2\dib{{+}}\dots\dib{{+}}m^2$. 
З~аналогічних причин, такий самий залишок % від ділення на~$m$ 
мають і сума усього третього проміжку $(2m{+}1)^2\dib{{+}}(2m{+}2)^2\dib{{+}}\dots\dib{{+}}(2m{+}m)^2$, і сума будь-якого подальшого. Тому досить цей однаковий для всіх проміжків залишок домножити на ${n\bdiv m}$, а\nolinebreak[3] потім, якщо % $n\bmod m\neq0$, 
$n$\nolinebreak[2] не\nolinebreak[3] кратне\nolinebreak[3] $m$, окремо порахувати й додати шматочок від ${(n\bdiv m){\cdot}m}\dib{{+}}1$ до~$n$ (або від\nolinebreak[2] 1\nolinebreak[2] до~$n\bmod m$). 

Таким чином, сумарна кількість ітерацій усіх циклів не\nolinebreak[3] перевищить $m\dib{{+}}(n\bmod m)\dib{{<}}2m\dib{{\<}}{2{\cdot}10^6}$. Це\nolinebreak[3] довше, ніж у <<способі\nolinebreak[3] $\No\,1$>>, але теж вкладається у\nolinebreak[2] 1~сек. Щоправда, якби обмеження було не\nolinebreak[3] ${m{\<}10^6}$, а\nolinebreak[2] ${m{\<}10^9}$, цей розв'язок став~би неможливим (а~<<спосіб\nolinebreak[3] $\No\,1$>> лишився~б можливим).

