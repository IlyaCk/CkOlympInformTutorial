\begin{problem}{Всюдисущі ч\'{и}сла}{Клавіатура (stdin)}{Екран (stdout)}{3 сек}{64 мегабайти}\label{sec:omnipresent-numbers}

Дано прямокутну таблицю $N{\*}M$\nolinebreak[3] чисел. Гарантовано, що\nolinebreak[2] у\nolinebreak[3] кожному окремо взятому рядку всі ч\'{и}сла різні й монотонно зростають.

Напишіть програму, яка шукатиме перелік (також у\nolinebreak[3] порядку зростання) всіх тих чисел, які зустрічаються\linebreak[1] в\nolinebreak[3] усіх $N$ рядках.

\InputFile	слід прочитати зі стандартного входу (клавіатури). У~першому рядку задано два числ\'{а} $N$ та~$M$. Далі йдуть $N$ рядків, кожен з яких містить рівно~$M$ розділених пропусками чисел (гарантовано у~порядку зростання). 

\OutputFile	виведіть на стандартний вихід (екран). Програма має вивести в один рядок через пробіли у порядку зростання всі ті числа, які зустрілися абсолютно в усіх рядках. Кількість чисел виводити не~треба. Після виведення всіх чисел потрібно зробити одне переведення рядка. Якщо нема жодного числа, що зустрілося в усіх рядках, виведення повинно не~містити жодного видимого символу, але містити переведення рядка.


\Example
\begin{exampleSimple}{6em}{3em}%
\exmp{4 5
6 8 10 13 19
8 9 13 16 19
6 8 12 13 15
3 8 13 17 19}{8 13}%
\end{exampleSimple}

\Scoring
20\% балів припадатиме на тести, в~яких $3{\<}{N,M}{\<}20$, значення чисел від~0 до~100.

Ще 20\% --- на тести, в~яких $3{\<}{N,M}{\<}20$, значення чисел від $-10^9$ до~$+10^9$.

Ще 20\% --- на тести, в~яких $1000{\<}{N,M}{\<}1234$, значення від 0 до 12345.

Решта 40\% --- на тести, в~яких $1000{\<}{N,M}{\<}1234$, значення від $-10^9$ до~$+10^9$.

Здавати потрібно одну програму, а~не~чотири; різні обмеження вказані, щоб пояснити, скільки балів можна отримати, розв’язавши задачу не~повністю.


\end{problem}
