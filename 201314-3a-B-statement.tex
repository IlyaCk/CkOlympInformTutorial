\begin{problemAllDefault}{Точні квадрати}

Напишіть програму, яка знаходитиме кількість натуральних чисел із проміжку $[a; b]$, які задовольняють одночасно двом таким вимогам:
\begin{enumerate}
\item
число є точним квадратом, тобто корінь з нього цілий (наприклад, точними квадратами є ${1{=}1^2}$, ${9{=}3^2}$, ${1024{=}32^2}$; а\nolinebreak[3] 8, 17, 1000 не~є точними квадратами).
\item
сума цифр цього числа кратна~$K$. (Наприклад, сума цифр числа 16 рівна ${1{+}6{=}7}$.)
\end{enumerate}
Програма повинна прочитати три числа в одному рядку $a$~$b$~$K$ і вивести одне число\nolinebreak[3] --- кількість чисел, які задовольняють умовам.

\Scoring	В усіх тестах виконується $1\dib{{\<}}a\dib{{\<}}b\dib{{\<}}2\cdot10^9$, $2\dib{{\<}}K\dib{{\<}}42$. 

40\%~балів припадає на тести, в~яких виконується $1\dib{{\<}}a\dib{{\<}}b\dib{{\<}}30\,000$, ${K{=}9}$.


\Example
\begin{exampleSimple}{5em}{3em}%
\exmp{7 222 9}{4}%
\end{exampleSimple}

\Note	Цими чотирма числами є 9, 36, 81, 144. 

\end{problemAllDefault}
