\begin{problem}{Дільники}{Клавіатура (stdin)}{Екран (stdout)}{2 сек}{64 мегабайти}

Для натурального числ\'{а}~$N$, виведіть у порядку зростання всі його різні натуральні дільники.

\InputFile	слід прочитати зі стандартного входу (клавіатури). Це буде єдине натуральне число\nolinebreak[3] $N$. $1\dib{{\<}}N\dib{{\<}}1234567891011$.

\OutputFile	--- послідовність усіх різних натуральних дільників, у порядку зростання — слід вивести на стандартний вихід (екран). Виводити обов’язково в один рядок, розділяючи пробілами.


\Examples
\begin{exampleSimple}{5em}{21em}%
\exmp{9}{1 3 9}%
\exmp{120}{1 2 3 4 5 6 8 10 12 15 20 24 30 40 60 120}%
\end{exampleSimple}

\Scoring	120 балів (з~250) припадатиме на тести, в яких $1\dib{{\<}}N\dib{{\<}}4321$.

Решта 130\nolinebreak[3] балів\nolinebreak[3] --- на тести, в яких $12345678\dib{{\<}}N\dib{{\<}}1234567891011$.

Здати потрібно одну програму, а не окремі для двох випадків; різні обмеження вводяться виключно для того, щоб дати приблизне уявлення, скільки балів можна отримати, розв’язавши задачу не\nolinebreak[3] повністю.

\end{problem}