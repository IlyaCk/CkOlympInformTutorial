\documentclass[14pt,a4paper]{extarticle}

% % % \usepackage{arial}
\usepackage{cmap}
\usepackage[T2A]{fontenc}
\usepackage[utf8]{inputenc}
\usepackage[english,russian,ukrainian]{babel}
\usepackage[metapost,truebbox,mplabels]{mfpic}
\usepackage{enumitem}
\usepackage{epigraph}
\usepackage{floatflt}
\usepackage{multicol}
\usepackage{longtable}
\usepackage{graphicx}
\usepackage{color}
\usepackage{textcomp}
\usepackage{amstext}
\usepackage{amsmath}
\usepackage{amssymb}
\usepackage{amsfonts}
\usepackage{verbatim}
\usepackage{alltt}

% \twocolumn

% \usepackage[landscape,russian]{olymppp}

\usepackage[ukrainian]{olymppp}

\usepackage{anysize}
\usepackage{listcorr}

\def\dib#1{\,#1\discretionary{}{\mbox{$#1$}}{}\,}

\marginsize{20mm}{20mm}{7mm}{20mm}


% \renewcommand{\baselinestretch}{1.28125}
\renewcommand{\baselinestretch}{1.3125}

\opengraphsfile{pics}

\begin{document}

\def\<{\leqslant}
\def\>{\geqslant}
\def\*{\times}
\def\dib#1{\,#1\discretionary{}{\mbox{$#1$}}{}\,}
\def\op{\mathop{\rm op}\nolimits}
\def\bdiv{\mathop{\rm div}\nolimits}
\def\opt{\mathop{\rm opt}}
\def\isdiv{\mathbin{\hbox to 0.25em{\hfill\hbox to 0 pt{\raisebox{0pt}{\hss$\cdot$\hss}}\hbox to 0 pt{\raisebox{-.6ex}{\hss$\cdot$\hss}}\hbox to 0 pt{\raisebox{.6ex}{\hss$\cdot$\hss}}\hspace{0.15em}\hfill}\,}}

\newlength{\myparindent}


\newlength{\mytemplen}
\newlength{\mytemplensecond}
\newlength{\mytemplenthird}
\newsavebox{\mypictbox}
\def\myrightfigure#1#2{%
\savebox{\mypictbox}{\noindent{}#2}%
\settowidth{\mytemplen}{\usebox{\mypictbox}}%
\settoheight{\mytemplenthird}{\usebox{\mypictbox}}%
\ifdim\mytemplen<0.8\textwidth%
\noindent%
\setlength{\mytemplensecond}{\textwidth}%
\addtolength{\mytemplensecond}{-\mytemplen}%
\addtolength{\mytemplen}{3pt}% ??? better to find alike standard len
\hspace*{\mytemplensecond}\usebox{\mypictbox}%
\par\vspace*{-0.5\baselineskip}\par%
\vspace*{-\mytemplenthird}
\vspace{-\parskip}
\hangindent=-\mytemplen
\hangafter=-#1
\else
\begin{center}
\usebox{\mypictbox}%
\par
\end{center}
% \vspace{-\baselineskip}
\fi%
}

\def\mytextandpicture#1#2{%
\setlength{\myparindent}{\parindent}%
\savebox{\mypictbox}{\noindent{}#2}%
\settowidth{\mytemplensecond}{\usebox{\mypictbox}}%
\setlength{\mytemplen}{\textwidth}%
\addtolength{\mytemplen}{-\mytemplensecond}%
\addtolength{\mytemplen}{-3mm}%
\noindent\mbox{}\hfill\parbox{\mytemplen}{\hspace*{\myparindent}#1}\hfill\hspace{2.5mm}\hfill\parbox{\mytemplensecond}{\usebox{\mypictbox}}\hfill\mbox{}\\
}

\def\myflfigaw#1{%
\savebox{\mypictbox}{\noindent{}#1}%
\settowidth{\mytemplen}{\usebox{\mypictbox}}%
% \addtolength{\mytemplen}{1mm}%
\ifdim\mytemplen<0.75\textwidth%
\begin{floatingfigure}[r]{\mytemplen}%
\mbox{\noindent%
\usebox{\mypictbox}%
\vspace{-6pt}}
\end{floatingfigure}%
\else
\begin{figure*}[h]%
\usebox{\mypictbox}%
\end{figure*}%
\fi%
}

\def\myhrulefill{\vspace{\baselineskip}\par\vspace*{-1.75\baselineskip}\par\hrulefill\par\vspace{-0.5\baselineskip}}

\newenvironment{problemAllDefault}[1]{\vspace{10mm}\par\begin{problem}{#1}{Клавіатура (stdin)}{Екран (stdout)}{1 сек}{64 мегабайти}}{\end{problem}}

\newif{\ifallIdeOneLinksCopiedHere}
\allIdeOneLinksCopiedHerefalse


%%% \in%put pseudo-title

% % % \begin{sffamily}

\begin{center}

\begin{huge}

~

\vfill

Цю сторінку замінити на сторінку з титулкою, набраною якимись іншими засобами

\vfill

~

\clearpage

~

\vfill

Цю сторінку замінити на сторінку з <<выходными сведениями>>, набраною якимись іншими засобами

\vfill

~

\end{huge}

\end{center}

\clearpage

\renewcommand{\thecontestname}{Олімпіади з інформатики (програмування)}
\renewcommand{\thecontestlocation}{Черкаська обл.}
\renewcommand{\thecontestdate}{2013--2015 роки}

\newcommand{\stdinOrInputTxt}{Або клавіатура, або input.txt}
\newcommand{\stdoutOrOutputTxt}{Або екран, або output.txt}


\tableofcontents


\input intro

\input FAQ

\section{Задачі та розбори}
\vspace{-0.5\baselineskip}

\input 201314-oioi

\input 201314-2

\input 201314-3a

\input 201415-oioi

\addtocontents{toc}{\protect\pagebreak\par}

\input 201415-2

\input 201415-3

% % % \clearpage

\renewcommand{\thecontestname}{Олімпіади з інформатики (програмування)}
\renewcommand{\thecontestlocation}{Черкаська обл.}
\renewcommand{\thecontestdate}{2013--2015 роки}

% % % \renewcommand{\baselinestretch}{0.875}

% % % \end{sffamily}

\end{document}