\Tutorial	
Задача складн\'{а} необхідністю дуже акуратно розглядати випадки, але проста тим, що потрібні \emph{лише} розгалуження та присвоєння.
% Задача досить складн\'{а} тим, що треба дуже акуратно розглядати випадки, і разом з тим дуже проста тим, що для остаточної реалізації програми не\nolinebreak[3] потрібні ніякі засоби, крім розгалужень та присвоєнь. 

Нехай змінні \texttt{n1}, \texttt{n2}, \texttt{n3}, \texttt{n4} та \texttt{n5} містять потрібні кількості дощечок розмірами $1{\*}1$, $2{\*}1$, $3{\*}1$, $4{\*}1$ та $5{\*}1$ відповідно, у змінній \texttt{res} будемо будувати відповідь задачі. Протягом роботи алгоритму значення деяких зі змінних \texttt{n1}, \texttt{n2}, \texttt{n3}, \texttt{n4} або \texttt{n5} можуть зменшуватися\nolinebreak[3] --- по мірі того, як враховуємо відповідні кількості у змінній \texttt{res}, яка наприкінці міститиме остаточну відповідь. 






{

\def\leftColumnWidth{0.35\textwidth}
\def\rightColumnWidth{0.6\textwidth}
\def\leftCell#1{\ttfamily\obeylines\obeyspaces\frenchspacing
\begin{minipage}[t]{\leftColumnWidth}
{\ttfamily\obeylines\obeyspaces\frenchspacing #1}
\end{minipage}}
\def\rightCell#1{
\begin{minipage}[t]{\rightColumnWidth}
{#1}
\end{minipage}\medskip}

\def\tabbb{\hspace*{2em}}

\begin{longtable}{|p{\leftColumnWidth}|p{\rightColumnWidth}|}
\hline
\multicolumn{1}{|c|}{Фрагмент коду} 
&
\multicolumn{1}{|c|}{Коментар}
\\\hline\endhead

\leftCell{res = n5}
&
\rightCell{(1)~На кожну дощечку $5{\*}1$ неминуче потрібна окрема ціла дощечка.}
\\\hline

\leftCell{res += n4}
&
\rightCell{(2)~На кожну дощечку $4{\*}1$ теж потрібна окрема дощечка\dots}
\\\hline

\leftCell{n1 -= min(n1,n4)}
&
\rightCell{(3)~\dots{}і кожен обрізок можна взяти як дощечку $1{\*}1$. Тому \emph{подальша} потреба в дощечках $1{\*}1$ складає вже не\nolinebreak[3] \texttt{n1}, а\nolinebreak[1] або\nolinebreak[2] \mbox{\texttt{n1–n4}}, або~\texttt{0}.}
\\\hline

\leftCell{res += n3}
&
\rightCell{(4)~На кожну дощечку розмірами $3{\*}1$ теж неминуче потрібна окрема дощечка\dots}
\\\hline

\leftCell{if n2 > n3:\\
\tabbb{}n2 -= n3}
&
\rightCell{(5)~\dots{}причому, при $N_2{>}N_3$ використовуємо кожен обрізок як дощечку $2{\*}1$\dots}
\\\hline

\leftCell{else:\\
\tabbb{}n3 -= n2\\
\tabbb{}n2 = 0\\
\tabbb{}n1 -= min(n3*2, n1)}
&
\rightCell{(6)~\dots{}а при $N_2{\<}N_3$ формуємо з обрізків \emph{усі} дощечки $2{\*}1$, а\nolinebreak[1] з\nolinebreak[1] решти ще й \texttt{\mbox{min(n3*2,}\nolinebreak[2] \mbox{n1)}} дощечок $1{\*}1$ (доки\nolinebreak[3] не\nolinebreak[1] закінчаться дощечки\nolinebreak[2] $3{\*}1$ або потреба у\nolinebreak[3] дощечках\nolinebreak[2] $1{\*}1$).}
\\\hline

\leftCell{res += n2//2\\
n1 -= min(n1,n2//2)\\
n2 \%= 2}
&
\rightCell{(7)~Оскільки всі дощечки $3{\*}1$ \emph{вже} сформовані, тепер на кожну пару дощечок $2{\*}1$ потрібна окрема дощечка, причому обрізок можна використати як дощечку $1{\*}1$.}
\\\hline

\leftCell{if n2==1:\\
\tabbb{}res += 1\\
\tabbb{}n1 -= min(n1,3)\\
\tabbb{}n2 = 0}
&
\rightCell{(8)~Якщо на попередньому кроці кількість дощечок $2{\*}1$ була непарна, то зараз треба сформувати останню дощечку $2{\*}1$, причому з\nolinebreak[3] обрізку можна зробити до~3 дощечок $1{\*}1$.}
\\\hline

\leftCell{res += n1//5\\
\tabbb{}n1 \%= 5}
&
\rightCell{(9)~Якщо після усіх попередніх кроків усе ще є потреба в дощечках $1{\*}1$, формуємо їх, розрізаючи кожну дощечку на 5 частин.}
\\\hline


\leftCell{if n1 > 0:\\
\tabbb{}res += 1}
&
\rightCell{(10)~Якщо після попереднього кроку все ще є потреба у дощечках $1{\*}1$, то вона $\<$4~штук, для чого достатньо ще\nolinebreak[2] \emph{однієї} дощечки.}
\\\hline
\end{longtable}

}

\vspace{-\baselineskip}

(Мова коду\nolinebreak[3] --- Python~3;\hspace{0.5em plus 1em}\linebreak[1]
``\verb"="''\nolinebreak[3] (одинарне)\nolinebreak[3] --- присвоєння;\hspace{0.5em plus 1em}\linebreak[1]
``\verb"=="''\nolinebreak[3] (подвійне)\nolinebreak[3] --- перевірити, чи\nolinebreak[3] дорівнює;\hspace{0.5em plus 1em}\linebreak[1]
``\verb"%"''\nolinebreak[3] --- залишок від ділення (\texttt{mod});\hspace{0.5em plus 1em}\linebreak[1]
``\verb"//"''\nolinebreak[3] (подвійне)\nolinebreak[3] --- \emph{цілочисельне} ділення (\texttt{div});\hspace{0.5em plus 1em}\linebreak[1]
``\verb"a+=b"''\nolinebreak[3] --- те\nolinebreak[2] само, що \verb"a=a+b", тобто до~\verb"a" додати~\verb"b" і покласти результат у\nolinebreak[1] ту\nolinebreak[2] саму змінну~\verb"a"; аналогічно ``\verb"a-=b"'', ``\verb"a%=b"''.)

Аргументація дій у\nolinebreak[3] коментарях зовсім\nolinebreak[1] не\nolinebreak[3] зайва
(див.\nolinebreak[2] також стор.~\pageref{text:need-or-no-need-to-prove}). Бо\nolinebreak[3] задача є частковим випадком \emph{задачі упаковки корзин} (\emph{bin\nolinebreak[1] packing problem}), для загального вигляду якої не\nolinebreak[3] відомо правильного швидкого алгоритма (про\nolinebreak[1] простий не\nolinebreak[3] йдеться; наука не~знає навіть складн\'{о}го, щоб був правильний та швидкий одночасно). Наприклад, \emph{якби} розміри початкових дощечок були не\nolinebreak[3] $5{\*}1$, а\nolinebreak[3] $7{\*}1$, і\nolinebreak[3] треба було сформувати 2~шт.\nolinebreak[3] $3{\*}1$ і 4~шт.\nolinebreak[3] $2{\*}1$, то\nolinebreak[3] треба було~б узяти дві стандартні $7{\*}1$ і розрізати кожну на $(3{\*}1)\dib{{+}}(2{\*}1)\dib{{+}}(2{\*}1)$, діючи \emph{всупереч} ідеї <<перш за\nolinebreak[3] все вибирати якнайбільші розміри>>.

Посилити аргументацію (провести більш строге доведення цих коментарів) можна приблизно так. На\nolinebreak[2] усіх кроках\nolinebreak[2] (3),\nolinebreak[1] \mbox{(5)--(8)} справедливо <<нема смислу викидати у\nolinebreak[3] відходи те, що можна використати; навіть якщо виявиться, що його можна узяти й пізніше, ми нічого не\nolinebreak[3] втрачаємо, узявши раніше>>. 
\label{text:proof-example-parket-1}
На\nolinebreak[3] кроках\nolinebreak[3] (5), \nolinebreak[1] (6), \nolinebreak[1] (8)\nolinebreak[3] --- <<де\nolinebreak[3] можна, варто віддавати перевагу $2{\*}1$ над $1{\*}1$, бо замість кожної $2{\*}1$ завжди можна зробити хоч\nolinebreak[1] дві\nolinebreak[3] $1{\*}1$, хоч\nolinebreak[1] одну, а\nolinebreak[3] зробити $2{\*}1$ замість двох $1{\*}1$ можна далеко не~завжди>>. На\nolinebreak[3] кроках\nolinebreak[2] (9),\nolinebreak[3] (10), формується залишок тих дощечок\nolinebreak[3] $1{\*}1$, які ніяк не\nolinebreak[3] могли бути сформовані раніше, бо,\nolinebreak[2] станом на початок кроку\nolinebreak[3] (9), \verb"n1>0" \emph{лише} якщо на усіх попередніх кроках усі дощечки використовувалися геть без відходів.

А для більших розмірів стандартної дощечки десь колись настає момент, коли узагальнення таких міркувань перестають бути правильними\dots