\Tutorial	Задача досить складна тим, що треба дуже акуратно розглянути всі випадки, і разом з тим дуже проста тим, що для остаточної реалізації програми не\nolinebreak[3] потрібні ніякі засоби, крім розгалужень та присвоєнь. 

Нехай змінні \texttt{n1}, \texttt{n2}, \texttt{n3}, \texttt{n4} та \texttt{n5} містять потрібні кількості дощечок розмірами $1{\*}1$, $2{\*}1$, $3{\*}1$, $4{\*}1$ та $5{\*}1$ відповідно, у змінній \texttt{res} будемо будувати відповідь задачі. Протягом роботи алгоритму значення деяких зі змінних \texttt{n1}, \texttt{n2}, \texttt{n3}, \texttt{n4} або \texttt{n5} можуть зменшуватися — по мірі того, як враховуємо відповідні кількості у змінній \texttt{res}, яка наприкінці міститиме остаточну відповідь. 






{

\def\leftColumnWidth{0.35\textwidth}
\def\rightColumnWidth{0.6\textwidth}
\def\leftCell#1{\ttfamily\obeylines\obeyspaces\frenchspacing
\begin{minipage}[t]{\leftColumnWidth}
{\ttfamily\obeylines\obeyspaces\frenchspacing #1}
\end{minipage}}
\def\rightCell#1{
\begin{minipage}[t]{\rightColumnWidth}
{#1}
\end{minipage}\medskip}

\def\tabbb{\hspace*{2em}}

\begin{longtable}{|p{\leftColumnWidth}|p{\rightColumnWidth}|}
\hline
\multicolumn{1}{|c|}{Фрагмент коду} 
&
\multicolumn{1}{|c|}{Коментар}
\\\hline\endhead

\leftCell{res = n5}
&
\rightCell{На кожну дощечку розмірами $5{\*}1$ неминуче потрібна окрема ціла дощечка.}
\\\hline

\leftCell{res += n4}
&
\rightCell{На кожну дощечку розмірами $4{\*}1$ теж неминуче потрібна окрема дощечка\dots}
\\\hline

\leftCell{n1 -= min(n1,n4)}
&
\rightCell{\dots{}але обрізки після відрізань дощечок розмірами $4{\*}1$ можна використати як дощечки розмірами $1{\*}1$. Тому надалі можна вважати, що потреба в дощечках $1{\*}1$ складає вже не \texttt{n1}, а або \texttt{n1–n4}, або~\texttt{0}.}
\\\hline

\leftCell{res += n3}
&
\rightCell{На кожну дощечку розмірами $3{\*}1$ теж неминуче потрібна окрема дощечка\dots}
\\\hline

\leftCell{if (n2>n3):\\
\tabbb{}n2 -= n3}
&
\rightCell{\dots{}причому, при $N_2{>}N_3$ просто використовуємо всі обрізки як дощечки розмірами $2{\*}1$\dots}
\\\hline

\leftCell{else:\\
\tabbb{}n3 -= n2\\
\tabbb{}n2 = 0\\
\tabbb{}n1 -= min(n3*2, n1)}
&
\rightCell{\dots{}а при $N_2{\<}N_3$ спочатку формуємо з цих обрізків абсолютно всі дощечки $2{\*}1$, а із решти — ще й min(n3*2, n1) дощечок $1{\*}1$.}
\\\hline

\leftCell{res += n2/2\\
\tabbb{}n1 -= min(n1,n2/2)\\
\tabbb{}n2 \%= 2}
&
\rightCell{Оскільки всі дощечки розміром $3{\*}1$ раніше вже сформовані, то тепер на кожну пару дощечок розміром $2{\*}1$ неминуче потрібна окрема дощечка, причому обрізок можна використати як дощечку $1{\*}1$.\\
\emph{Ділення тут цілочисельне (div).}}
\\\hline

\leftCell{if n2==1:\\
\tabbb{}res += 1\\
\tabbb{}n1 -= min(n1,3)\\
\tabbb{}n2 = 0}
&
\rightCell{Якщо при виконанні попереднього етапу кількість дощечок розміром $2{\*}1$ була непарна, то зараз треба сформувати останню дощечку розміром $2{\*}1$, причому обрізок можна використати для формування від нуля до трьох дощечок розмірами $1{\*}1$.}
\\\hline

\leftCell{res += n1/5\\
\tabbb{}n1 \%= 5}
&
\rightCell{Якщо, незважаючи на усі попередні кроки, досі є потреба в дощечках розміром $1{\*}1$, формуємо їх, розрізаючи кожну дощечку на 5 частин.\\
\emph{Ділення тут цілочисельне (div).}}
\\\hline


\leftCell{if n1 > 0:\\
\tabbb{}res += 1}
&
\rightCell{Якщо після попереднього кроку все ще залишилася потреба у дощечках розміром $1{\*}1$ (від 1 до 4 штук), дя цього достатньо ще\nolinebreak[2] \emph{однієї} дощечки $5{\*}1$.}
\\\hline
\end{longtable}



}

(Мова коду\nolinebreak[3] --- Python;\hspace{0.5em plus 1em}\linebreak[1]
``\verb"="'' (одинарне)\nolinebreak[3] --- присвоєння;\hspace{0.5em plus 1em}\linebreak[1]
``\verb"=="'' (подвійне)\nolinebreak[3] --- перевірити, чи\nolinebreak[3] дорівнює;\hspace{0.5em plus 1em}\linebreak[1]
``\verb"%"''\nolinebreak[3] --- залишок від ділення (\texttt{mod});\hspace{0.5em plus 1em}\linebreak[1]
``\verb"a+=b"''\nolinebreak[3] --- те\nolinebreak[2] само, що \verb"a=a+b", тобто до старого значення~\verb"a" додати~\verb"b" і покласти результат у\nolinebreak[1] ту\nolinebreak[2] саму змінну~\verb"a"; аналогічно ``\verb"a-=b"'', ``\verb"a%=b"''.)
