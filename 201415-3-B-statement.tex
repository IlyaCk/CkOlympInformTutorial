\begin{problemAllDefault}{Прямокутники --- rectangles}

Дані два прямокутники, сторони яких паралельні вісям координат.

\Task Напишіть програму \texttt{rectangles}, яка б визначала площу їхнього об’єднання, тобто усієї частини площини, що покрита хоча б одним з прямокутників. Площу спільної частини обох прямокутників (якщо така є) слід враховувати один раз.

\InputFile містять два рядки, кожен з яких описує один з прямокутників, у форматі $x_{\min}$ $x_{\max}$ $y_{\min}$ $y_{\max}$. Усі координати є цілими числами, що не перевищують по модулю~1000.

\OutputFile Ваша програма має вивести єдине ціле число --- знайдену площу об’єднання.

\Examples
\vspace*{-\baselineskip}
\begin{exampleSimpleThree}{6em}{4em}{6em}{}%
\exmp{0 10 0 10
20 30 20 30}{200}{}%
\exmp{0 20 0 30
10 12 17 23}{600}{}%
\exmp{0 4 0 3
-2 1 2 4}{17}{\raisebox{-32pt}{\begin{mfpic}[9]{-2.3}{4.9}{-0.3}{4.9}
\axes
\dotted\lines{(-1,-0.1),(-1,4.1)}
\dotted\lines{(-2,-0.1),(-2,4.1)}
\dotted\lines{( 1,-0.1),( 1,4.1)}
\dotted\lines{( 2,-0.1),( 2,4.1)}
\dotted\lines{( 3,-0.1),( 3,4.1)}
\dotted\lines{( 4,-0.1),( 4,4.1)}
\dotted\lines{(-2.1, 1),(4.1, 1)}
\dotted\lines{(-2.1, 2),(4.1, 2)}
\dotted\lines{(-2.1, 3),(4.1, 3)}
\dotted\lines{(-2.1, 4),(4.1, 4)}
\rhatch\polygon{(0,0),(4,0),(4,3),(1,3),(1,4),(-2,4),(-2,2),(0,2)}
\pen{2pt}
\rect{(0,0),(4,3)}
\rect{(-2,2),(1,4)}
\end{mfpic}}}%
\end{exampleSimpleThree}

% % % Зображення останнього з прикладів:


\end{problemAllDefault}